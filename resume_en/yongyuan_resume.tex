%%%%%%%%%%%%%%%%%%%%%%%%%%%%%%%%%%%%%%%%%
% Medium Length Professional CV
% LaTeX Template
% Version 2.0 (8/5/13)
%
% This template has been downloaded from:
% http://www.LaTeXTemplates.com
%
% Original author:
% Trey Hunner (http://www.treyhunner.com/)
%
% Important note:
% This template requires the resume.cls file to be in the same directory as the
% .tex file. The resume.cls file provides the resume style used for structuring the
% document.
%
%%%%%%%%%%%%%%%%%%%%%%%%%%%%%%%%%%%%%%%%%

%----------------------------------------------------------------------------------------
%	PACKAGES AND OTHER DOCUMENT CONFIGURATIONS
%----------------------------------------------------------------------------------------

\documentclass{resume} % Use the custom resume.cls style

\usepackage[left=0.75in, top=0.3in, right=0.75in, bottom=0.5in]{geometry} % Document margins
%\usepackage[left=0.75in, top=0.5in, right=0.75in, bottom=0.6in]{geometry} % Document margins

% 设置最后一次更新时间
\usepackage{lastpage}
\usepackage{fancyhdr}
\pagestyle{fancy}
\fancyhf{}
\fancyfoot[L]{\footnotesize\textit{Yong Yuan's resume MacVersion}}
\fancyfoot[R]{\footnotesize\textit{Last update: \today}}

\name{Yong Yuan} % Your name
%\address{34418 Bacon Place \\ Fremont, CA 12345} % Your address
%\address{123 Pleasant Lane \\ City, State 12345} % Your secondary addess (optional)
\address{(086)~$\cdot$~150~$\cdot$~2955~$\cdot$~2208 \\ willard.yuan@gmail.com \\ https://github.com/willard-yuan}
%Your phone number and email
\begin{document}

%----------------------------------------------------------------------------------------
%	EDUCATION SECTION
%----------------------------------------------------------------------------------------

\begin{rSection}{Education}
{\bf University of Chinese Academy of Sciences} \hfill {\em Aug. 2013 - Jun. 2016} \\
M.S. in  Signal \& Information Processing (Research in image retrieval)\\
{\bf Xidian University} \hfill {\em Aug. 2009- Jun. 2013} \\
B.S. in Computer Engineering
\end{rSection}

%----------------------------------------------------------------------------------------
%	WORK EXPERIENCE SECTION
%----------------------------------------------------------------------------------------

\begin{rSection}{Professional Experience}

\begin{rSubsection}{Center for OPTical IMagery Analysis and Learning (OPTIMAL)}{\em Aug. 2013 - Present}{Graduate Researcher}{XI'AN, CN}
\item Designed content based image retrieval algorithm to improve image search accuracy and efficiency, and used Matlab or C++ to build image retrieval prototype.
\item Mastered bag of virtual words (BoVW) model, Vector of locally aggregated descriptors (VLAD), fisher vector (FV) for image representation, and got familar with Convolutional neural network (CNN), the performance evaluation of image retrieval and other machine learning algorithms.
\item Proposed two new hashing based methods for approximate nearest neighbor search. One is based on sparse reconstruction to learn hashing functions and has been published. Another based on matrix factorization has been fully written and will be submitted.
\item Developed a Matlab toolkit box for someone interested in designing hashing method. The toolkit box contains several popular hashing methods and various evaluations to validate performance are included.
\item Participated in contest of clothes and shoes (150,000 images respectly) images retrieval, and accumulated much experience in duplicate image search and object retrieval.
\end{rSubsection}

\end{rSection}

%----------------------------------------------------------------------------------------
%	COURSE PROJECTS
%----------------------------------------------------------------------------------------

\begin{rSection}{ACADEMIC PROJECTS}

\begin{pSubsection}{DuplicateSearch}{Graduate Researcher}{Mar. 2015 - Present}
{DupSearch is a image retrieval prototype for duplicate search or object retrieval. It's a project based on BoVW or FV model and I developed it indepently.}
\item Extracted SIFT feature descriptor to overcome brigthness variance, rotation variance, translation variance and scale variance.
\item Quantized each extracted local feature into one of visual words, and represented each image with a global feature by histogram of the visual words.
\item Built multithreads by openMP to speed up feature extraction and the process of clustering.
\item Improved the mean average precision (mAP) by reranking algorithm. On Oxford Building public database the mAP reached 84.89\% after reranking with 500,000 virtual words.
\item Tested the prototype on two large datasets including 150,000 clothes images and 130,000+ logo images, and optimized the search time and accuracy.
\end{pSubsection}
\vspace{-0.5em}

%------------------------------------------------

\begin{pSubsection}{PicSearch Web Application}{Graduate Researcher}{Jan. 2015 - Apr. 2015}
{PicSearch is an online image retrieval demo which uses the vgg CNN model. I developed it for fun and learning the deep learning architect.}
\item Completed feature extraction offline based on the model pre-trained on imageNet with Matlab.
\item Made dimensionality reduction by PCA to reduce memory usage (the memory is limited to 1G) and speed up the query response time.
\item Deployed the prototype with the lightweight web development framework CherryPy on CentOS. The front-end interface was based on Boostrap framework.
\item Optimized the code to make sure it respond to the user's query (milliseconds) in a timely manner. The image dataset I used is Caltech-256 public dataset (29,780 images) and the online demo address of PicSearch is: search.yongyuan.name. The demo has no upload function yet.
\end{pSubsection}
\vspace{-0.5em}

%------------------------------------------------

\begin{pSubsection}{Hashing Baseline for Image Retrieval}{Graduate Researcher}{Jan. 2014 - Apr. 2015}
{HABIR is a hashing baseline Matlab toolbox for image retrieval. The toolbox was developed to build a baseline for evaluting a designed hashing method. It's very easy to use and contains nice state-of-the-art hashing methods. Various evaluations are included.
}
\item Designed a reasonable structure to make sure the toolbox is extended conveniently if a new hashing method is added, and it was very easy to read the souce codes.
\item Implemented some hashing methods, note some hashing methods were found at the authors homepages, and inserted them to the framwork.
\item Implemented some evaluations for validating the effectiveness of hashing methods, and made the evaluations complete.
\item Tested the developed toolbox on several public dataset to verify the correctness of implemention.
\item Designed and proposed new hashing methods for image retrieval based on the toolbox.
\end{pSubsection}
\end{rSection}

%-------------------------------------------
%COURSE WORK
%-------------------------------------------

%\begin{rSection}{Relevant Coursework}
%\begin{rSubsection}{University of California San Diego}{}{}{}
%\item CSE150 : Artificial Intelligence : A Statistical Approach
%\item CSE120 : Principles of Operating Systems
%\item CSE110 : Software Engineering
%\item CSE105 : Automata and Computability Theory
%\item CSE103 : Probability and Statistics
%\item CSE101 : Design and Analysis of Algorithms
%\end{rSubsection}
%\begin{rSubsection}{University of California Santa Cruz}{}{}{}
%\item CMPE120 : Microprocessor System Design
%\item CMPS101 : Algorithms and Abstract Data Types
%\item CMPE100 : Logic Design
%\item EE101 : Introduction to Electronic Circuits
%\item EE103 : Signals and Systems
%\end{rSubsection}
%\end{rSection}
%----------------------------------------------------------------------------------------
%	VOLUNTEER
%----------------------------------------------------------------------------------------

%\begin{rSection}{Volunteer Experience}
%\item Tennis Instructor at Fremont Tennis Center
%\item 5th Grade Tutor
%\end{rSection}

%----------------------------------------------------------------------------------------
%	TECHNICAL STRENGTHS SECTION
%----------------------------------------------------------------------------------------
\begin{rSection}{Technical Skills}
\begin{tabular}{ @{} >{\bfseries}l @{\hspace{6ex}} l }
Computer Languages & Python, Matlab, C++/C, HTML, Javascript, CSS\\
Machine Learning & BoVW, VLAD, Hashing, SVM, FV, CNN, KD-tree\\
Tools & Git, Chrome, OpenCV, Django\\
Operating Systems & OS X, Linux, Windows\\\\
\end{tabular}
\vspace{-2em}

%----------------------------------------------------------------------------------------
%	AWARDS
%----------------------------------------------------------------------------------------
\begin{rSection}{Awards}
\item Best Paper Runner-up Award(2014)
\item National Scholarship(2012)
\item The First Prize Scholarship(2010)
\item National Scholarship for Encouragement(2009)
\end{rSection}
\end{rSection}

%----------------------------------------------------------------------------------------
%	EXAMPLE SECTION
%----------------------------------------------------------------------------------------

%\begin{rSection}{Section Name}

%Section content\ldots

%\end{rSection}

%----------------------------------------------------------------------------------------

\end{document}