%%%%%%%%%%%%%%%%%%%%%%%%%%%%%%%%%%%%%%%%%
% Medium Length Professional CV
% LaTeX Template
% Version 2.0 (8/5/13)
%
% This template has been downloaded from:
% http://www.LaTeXTemplates.com
%
% Original author:
% Trey Hunner (http://www.treyhunner.com/)
%
% Important note:
% This template requires the resume.cls file to be in the same directory as the
% .tex file. The resume.cls file provides the resume style used for structuring the
% document.
%
%%%%%%%%%%%%%%%%%%%%%%%%%%%%%%%%%%%%%%%%%

%----------------------------------------------------------------------------------------
%	PACKAGES AND OTHER DOCUMENT CONFIGURATIONS
%----------------------------------------------------------------------------------------

\documentclass{resume} % Use the custom resume.cls style

\usepackage[left=0.75in, top=0.3in, right=0.75in, bottom=0.4in]{geometry} % Document margins
%\usepackage[left=0.75in, top=0.5in, right=0.75in, bottom=0.6in]{geometry} % Document margins

\name{Yong Yuan} % Your name
%\address{34418 Bacon Place \\ Fremont, CA 12345} % Your address
%\address{123 Pleasant Lane \\ City, State 12345} % Your secondary addess (optional)
\address{(510)~$\cdot$~456~$\cdot$~8972 \\ willard.yuan@gmail.com \\ https://github.com/willard-yuan}
%Your phone number and email
\begin{document}

%----------------------------------------------------------------------------------------
%	EDUCATION SECTION
%----------------------------------------------------------------------------------------

\begin{rSection}{Education}
{\bf University of Chinese Academy of Sciences} \hfill {\em Aug. 2013 - Jun. 2016} \\
M.S. in  Signal \& Information Processing (Research in image retrieval)\\
{\bf Xidian University} \hfill {\em Aug. 2009- Jun. 2013} \\
B.S. in Computer Engineering
\end{rSection}

%----------------------------------------------------------------------------------------
%	WORK EXPERIENCE SECTION
%----------------------------------------------------------------------------------------

\begin{rSection}{Experience}

\begin{rSubsection}{Center for OPTical IMagery Analysis and Learning (OPTIMAL)}{\em Aug. 2013 - Present}{Graduate Researcher}{XI'AN, CN}
\item Its Retrieval CBIR proficient performance evaluation, mastered BoW bag of words model, SIFT / SURF, VLAD characteristics descriptor.
\item Through continuous learning and accumulation master machine learning some commonly used means of dimensionality reduction, clustering algorithms, image classification method and image object recognition technology.
\item Studied in depth based on a hash of a large-scale image retrieval technology, familiar with the hash method more popular in recent years. For some popular and classical hashing method was performance testing and evaluation indicators, see HABIR toolkit homepage and hash coding method is proposed based on sparse expression, published in the ICIMCS14, another to write an article on the new hash article to be cast.
\item Participated in the contest pkbigdata on image retrieval, in clothes, shoes and other large image library (150,000) do with the money accumulated more experience in search; there is an image of a particular class, such as leather, textile and other image search experience; at 13 done on the order of ten thousand advertising logo Gallery search.
\end{rSubsection}

\end{rSection}

%----------------------------------------------------------------------------------------
%	COURSE PROJECTS
%----------------------------------------------------------------------------------------

\begin{rSection}{PROJECTS}

\begin{pSubsection}{DuplicateSearch}{Graduate Researcher}{Mar. 2015 - Present}
\item DupSearch is a duplicate search for object retrieval or written image retrieval prototype system has great value.
\item On average oxford building public database retrieval accuracy of 83.35\%, for light, rotation, perspective and so has good adaptability, online match on the server can respond to queries faster and without complicating the situation still existing model there are improvements to improve mAP space.
\item Image size of up to 150,000 test library, you can get a very good search results, the algorithm prototype system has been sold to a company, 150,000 clothes library retrieval results detailed in GitHub, in addition, for the search advertising logo can achieve high search accuracy.
\end{pSubsection}
\vspace{-0.5em}

%------------------------------------------------

\begin{pSubsection}{PicSearch  Web Application}{Graduate Researcher}{Jan. 2015 - Apr. 2015}
\item PicSearch is an online image retrieval prototype system that uses convolution to the CNN network model, we can achieve very satisfactory search results.
\item Complete the line image feature extraction, and made some dimensionality reduction, feature matching and sorting line background with python implementation python server using lightweight web development framework CherryPy, using front-end interface optimized Boostrap framework.
\item Library containing 29,780 images Caltech-256 public data sets, using the characteristics of memory resident mode optimized code, so that it can respond to the user's query (milliseconds) in a timely manner, the online demo Address PicSearch: search.yongyuan .name
\end{pSubsection}
\vspace{-0.5em}

%------------------------------------------------

\begin{pSubsection}{Hashing Baseline for Image Retrieval}{Graduate Researcher}{Jan. 2014 - Apr. 2014}
\item I do the image retrieval research in early 2013. My mission is to design efficient hashing algorithm to map the semantic similar images to the similar codes. There are two main advantages using hashing method for image retrieval, i.e. storage and computation efficience. To let more researchers focus on design hashing algorithm, I have built a hashing baseline, hoping this project can do some help for some researchers.
\end{pSubsection}
\end{rSection}

%-------------------------------------------
%COURSE WORK
%-------------------------------------------

%\begin{rSection}{Relevant Coursework}
%\begin{rSubsection}{University of California San Diego}{}{}{}
%\item CSE150 : Artificial Intelligence : A Statistical Approach
%\item CSE120 : Principles of Operating Systems
%\item CSE110 : Software Engineering
%\item CSE105 : Automata and Computability Theory
%\item CSE103 : Probability and Statistics
%\item CSE101 : Design and Analysis of Algorithms
%\end{rSubsection}
%\begin{rSubsection}{University of California Santa Cruz}{}{}{}
%\item CMPE120 : Microprocessor System Design
%\item CMPS101 : Algorithms and Abstract Data Types
%\item CMPE100 : Logic Design
%\item EE101 : Introduction to Electronic Circuits
%\item EE103 : Signals and Systems
%\end{rSubsection}
%\end{rSection}
%----------------------------------------------------------------------------------------
%	VOLUNTEER
%----------------------------------------------------------------------------------------

%\begin{rSection}{Volunteer Experience}
%\item Tennis Instructor at Fremont Tennis Center
%\item 5th Grade Tutor
%\end{rSection}

%----------------------------------------------------------------------------------------
%	TECHNICAL STRENGTHS SECTION
%----------------------------------------------------------------------------------------
\begin{rSection}{Technical Strengths}
\begin{tabular}{ @{} >{\bfseries}l @{\hspace{6ex}} l }
Computer Languages & Python, Matlab, C ++ / C, HTML, CSS\\
Machine Learning & CBIR technology proficient, proficient SVM, BoF, ANN, hash and other common machine learning methods to understand the depth of learning model \\
Tools & Git, Chrome, OpenCV, Python web development framework Django\\
Operating Systems & OS X, Linux\\\\
\end{tabular}

%----------------------------------------------------------------------------------------
%	AWARDS
%----------------------------------------------------------------------------------------
\begin{rSection}{Awards}
\item Best Paper Runner-up Award(2014)
\item National Scholarship(2012)
\item The First Prize Scholarship(2010)
\item National Scholarship for Encouragement(2009)
\end{rSection}
\end{rSection}

%----------------------------------------------------------------------------------------
%	EXAMPLE SECTION
%----------------------------------------------------------------------------------------

%\begin{rSection}{Section Name}

%Section content\ldots

%\end{rSection}

%----------------------------------------------------------------------------------------

\end{document}
