%% start of file `template.tex'.
%% Copyright 2006-2010 Xavier Danaux (xdanaux@gmail.com).
%
% This work may be distributed and/or modified under the
% conditions of the LaTeX Project Public License version 1.3c,
% available at http://www.latex-project.org/lppl/.


\documentclass[11pt,a4paper]{moderncv}

% moderncv themes
%\moderncvtheme[blue]{casual}                 % optional argument are 'blue' (default), 'orange', 'red', 'green', 'grey' and 'roman' (for roman fonts, instead of sans serif fonts)
\moderncvstyle{classic}                        % 选项参数是 ‘casual’, ‘classic’, ‘oldstyle’ 和 ’banking’
\moderncvcolor{blue}

\usepackage[utf8]{inputenc}                   % 替换你正在使用的编码
%\usepackage{CJKutf8}

% character encoding
%\usepackage[utf8]{inputenc}                   % replace by the encoding you are using
\usepackage[noindent]{ctex}

% adjust the page margins
\usepackage[scale=0.8]{geometry}
\setlength{\hintscolumnwidth}{3cm}						% if you want to change the width of the column with the dates
%\AtBeginDocument{\setlength{\maketitlenamewidth}{6cm}}  % only for the classic theme, if you want to change the width of your name placeholder (to leave more space for your address details
%\AtBeginDocument{\recomputelengths}                     % required when changes are made to page layout lengths

% personal data
\firstname{}
\familyname{袁勇}
%\title{个人简历}                              % optional, remove the line if not wanted
\address{}{西安市高新区}                       % optional, remove the line if not wanted
\mobile{150-2955-2208}                         % optional, remove the line if not wanted
%\phone{phone (optional)}                      % optional, remove the line if not wanted
%\fax{fax (optional)}                          % optional, remove the line if not wanted
\email{willard.yuan@gmail.com}                 % optional, remove the line if not wanted
\homepage{yongyuan.name}                       % optional, remove the line if not wanted
%\extrainfo{♂34岁}                             % optional, remove the line if not wanted
\photo[64pt]{qr}                               % '64pt' is the height the picture must be resized to and 'picture' is the name of the picture file; optional, remove the line if not wanted
%\quote{I'm normally not a praying man, but if you're up there, please save me, Superman}                 % optional, remove the line if not wanted

% to show numerical labels in the bibliography; only useful if you make citations in your resume
%\makeatletter
%\renewcommand*{\bibliographyitemlabel}{\@biblabel{\arabic{enumiv}}}
%\makeatother

% bibliography with mutiple entries
%\usepackage{multibib}
%\newcites{book,misc}{{Books},{Others}}

%\nopagenumbers{}                             % uncomment to suppress automatic page numbering for CVs longer than one page
%----------------------------------------------------------------------------------
%            content
%----------------------------------------------------------------------------------
\begin{document}
\maketitle

\section{学历}
\cventry{2013.09--至今}{硕士}{}{中国科学院大学}{信号与信息专业(保研)}{}  % arguments 3 to 6 can be left empty
\cventry{2009.09--2013.07}{学士}{}{西安电子科技大学}{电子信息科学与技术专业(专业top 3\%)}{}
\section{出版物}
\cventry{2014.04}{\textbf{Yong Yuan}\textnormal{, Xiaoqiang Lu, and Xuelong Li. Learning Hash Functions Using Sparse Reconstruction. in ICIMCS, 2014 (Best Paper Runner-up Award)}}{}{}{}{}
\cventry{2014.06}{\textnormal{朱文涛,}\textbf{袁勇}\textnormal{. Python计算机视觉编程(译),图灵出版社(书后半部分由我译)}}{}{}{}{}
\cventry{2015.02}{\textnormal{李学龙,卢孝强,}\textbf{袁勇}\textnormal{. 基于潜在语义最小哈希的图像检索的方法(专利)}}{}{}{}{}
\section{科研经历}
\subsection{中国科学院西安光学精密机械研究所(2013--至今)}
\cventry{2013--至今}{基于内容的图像检索(CBIR)技术}{}{课题研究方向}{}{硕士研究方向,致力于object retrieval(duplicate search)与相似语义图像搜索,技术层面关注低计算复杂度与查询的快速相应,并尽可能的提高图像检索的平均检索精度.
\begin{itemize}
\item 熟悉CBIR技术及其检索性能指标评价,掌握了BoVW词袋模型、倒排索引、相关反馈等检索技术,熟悉机器学习中一些常用聚类算法、分类方法以及物体识别技术.
\item 熟悉近几年来比较流行的哈希方法,针对一些流行的无监督哈希方法进行了总结与归纳,并将其整理成了HABIR哈希图像检索工具包.
\item
    比较深入地研究过哈希索引技术,提出了一种基于稀疏表达的哈希编码方法并发表于ICIMCS14上.
\item
    参加过pkbigdata上的图像检索大赛,有皮革图像检索的经历,在图像搜索方面积累了一定的经验.
\end{itemize}}
\cventry{2014.07--2015.05}{复杂低空飞行的自主避险理论与方法研究(973)}{}{项目参与者}{}{对复杂低空环境中可能的危险障碍物进行实时检测,并完成飞行器的自主避险.
\begin{itemize}
\item 负责可见光传感器数据与激光雷达传感器点云数据的融合,消除高压线检测时的误检.
\item 负责桥梁、高压线塔、作为异常目标入侵的滑翔机等危险障碍物的实时检测.
%\item 使用了opencv、dlib等计算机视觉开源库,非电力线类障碍物检测采用HOG+SVM物体检测方法.
\end{itemize}}
\cventry{2015.01--2015.04}{基于卷积神经网络的CBIR原型系统PicSearch}{}{兴趣驱动型项目}{}{PicSearch是一个在线图像检索演示系统,使用了在imageNet上已训练好的CNN模型提取图像特征,对于相似语义图像搜索能取得非常满意的检索效果.
\begin{itemize}
\item 线下完成图像特征的提取,在线特征匹配与排序用python实现,服务器采用了基于python的web开发框架CherryPy,前端用了Boostrap框架.
\item 图库为包含29780张图片的Caltech-256公开数据集,代码经过一定的优化后,能及时地响应用户的查询请求(毫秒级),在线演示地址: \link[search.yongyuan.name]{http://search.yongyuan.name/}.
\end{itemize}}
\cventry{2015.03--至今}{基于词袋模型的物体检索原型DupSearch}{}{兴趣驱动型项目}{}{DupSearch是针对object retrieval或duplicate search而写的CBIR算法原型系统,在15万图像库上测试获得了不错的检索效果,原型系统已被某公司买下.
\begin{itemize}
\item 在oxford building数据库上平均检索精度达到83.35\%, 在不复杂化现有模型情况下仍有改进提高mAP的空间.
\item 在15万幅衣服图像库上,对三张查询图片搜索的结果地址: \link[pkbigdata-image-search]{https://github.com/willard-yuan/pkbigdata-image-search}.
\end{itemize}}
\section{开源项目}
\cventry{2013.02--Now}{HABIR}{\link[github.com/willard-yuan/hashing-baseline-for-image-retrieval]{https://github.com/willard-yuan/hashing-baseline-for-image-retrieval}}{项目主页: \link[habir]{http://yongyuan.name/habir/}}{整理并实现一些流行典型的哈希算法及多种指标评价,目前该Matlab工具包已更新至V2.0}{}
\cventry{2013.12--Now}{pcv-book-code}{\link[github.com/willard-yuan/pcv-book-code]{https://github.com/willard-yuan/pcv-book-code/}}{项目主页: \link[Python计算机视觉编程]{http://yongyuan.name/pcvwithpython/}}{翻译《Programming Computer Vision with Python》时,为使读者更易于理解书中的内容,重新对书上的代码做了整理,并放在github上}{}
\cventry{2014.02--2014.05}{SRH}{\link[github.com/willard-yuan/sparse-reconstruction-hashing]{https://github.com/willard-yuan/sparse-reconstruction-hashing}}{基于稀疏重构的哈希编码方法的matlab实现及检索指标评价}{}{}
\section{IT技能}
\cvline{语言}{\small 掌握Python, Matlab, 熟悉C/C++, OpenCV, Linux, Django, HTML, CSS, Latex.}
\cvline{GitHub}{{\link[github.com/willard-yuan]{https://github.com/willard-yuan}}, 活跃.}
%\cvline{工具}{\small Linux, Git, Django, Latex, Markdown, Jekyll}
%\cvline{其他}{\small UML, OSGI, OOD, SOA, 设计模式, RUP/SCRUM 开发模型}
\section{奖项}
%\cventry{2014.07}{\textnormal {ICIMCS14最佳论文提名奖}}{}{}{}{}
%\cventry{2011--2012}{\textnormal {国家奖学金}}{}{}{}{}
%\cventry{2010--2011}{\textnormal {校内一等奖学金}}{}{}{}{}
%\cventry{2009--2010}{\textnormal {国家励志奖学金}}{}{}{}{}
\cvlistdoubleitem{Best Paper Runner-up Award(2014)}{国家奖学金(2012)}
\cvlistdoubleitem{校内一等奖学金(2011)}{国家励志奖学金(2010)}
\section{语言}
\cvline{英语}{\small CET-6和\small CET-4,具备阅读专业英文文献及写作能力.}
%\cvlanguage{英语}{具备阅读专业英文文献能力及写作能力}{CET-6}
\section{其他}
\cvline{}{靠谱,责任心强,乐于助人;热爱计算机视觉与互联网;轻度代码洁癖;具有较好的人际沟通、协调和组织能力.}
%\cvlistitem{}

% Publications from a BibTeX file without multibib\renewcommand*{\bibliographyitemlabel}{\@biblabel{\arabic{enumiv}}}% for BibTeX numerical labels
%\nocite{*}
%\bibliographystyle{plain}
%\bibliography{publications}       % 'publications' is the name of a BibTeX file

% Publications from a BibTeX file using the multibib package
%\section{Publications}
%\nocitebook{book1,book2}
%\bibliographystylebook{plain}
%\bibliographybook{publications}   % 'publications' is the name of a BibTeX file
%\nocitemisc{misc1,misc2,misc3}
%\bibliographystylemisc{plain}
%\bibliographymisc{publications}   % 'publications' is the name of a BibTeX file

\end{document}


%% end of file `template_en.tex'.
