%!TEX program = xelatex
\documentclass[11pt,a4paper]{moderncv}

%主题设置
\moderncvstyle{classic}                        % 选项参数是 ‘casual’, ‘classic’, ‘oldstyle’ 和 ’banking’
\moderncvcolor{blue} % optional argument are 'blue' (default), 'orange', 'red', 'green', 'grey' and 'roman' (for roman fonts, instead of sans serif fonts)

\usepackage{fontspec,xunicode,xltxtra}
% 设置英文字体
\defaultfontfeatures{Scale=MatchLowercase} % 设置字体大小,Scale=0.8等
\setmainfont{Hiragino Sans GB} %主字体
\setsansfont{Hiragino Sans GB} %无衬线字体
\setmonofont{Hiragino Sans GB} %等宽字体
% 设置中文字体
\usepackage{xeCJK}
% Scheme: {STFangsong}, 冬青黑体简体中文{Hiragino Sans GB W3}, 
\setCJKmainfont{Hiragino Sans GB W3}
\setCJKsansfont{Hiragino Sans GB W3}
\setCJKmonofont{Hiragino Sans GB W3}

% 设置最后一次更新时间
\usepackage{lastpage}
\usepackage{fancyhdr}
\pagestyle{fancy}
\fancyhf{}
\fancyfoot[L]{\footnotesize\textit{袁勇个人简历MacVersion}}
\fancyfoot[R]{\footnotesize\textit{最后一次更新于: \today}}

% 调整页边距
\usepackage[scale=0.85]{geometry}
\setlength{\hintscolumnwidth}{3cm}						% if you want to change the width of the column with the dates
%\AtBeginDocument{\setlength{\maketitlenamewidth}{6cm}}  % only for the classic theme, if you want to change the width of your name placeholder (to leave more space for your address details
%\AtBeginDocument{\recomputelengths}                     % required when changes are made to page layout lengths

% 在其他左侧插入标志
\usepackage{manfnt}
\newcommand{\hello}{{\tiny\textdbend}}

%\usepackage{xcolor}
\linespread{1.2}

% 设置超链接字体颜色
\AtBeginDocument{
    \hypersetup{colorlinks,urlcolor=blue}
}
%\hypersetup{%
%  colorlinks=true,% hyperlinks will be black
%  linkbordercolor=red,% hyperlink borders will be red
%  pdfborderstyle={/S/U/W 1}% border style will be underline of width 1pt
%}

% 设置个人信息
%\firstname{}
%\familyname{袁勇}
\name{\textbf{袁}}{\textbf{勇}}
\title{\textbf{湖北/男}}                              % optional, remove the line if not wanted
\address{}{西安市高新区}                       % optional, remove the line if not wanted
\mobile{150-2955-2208}                         % optional, remove the line if not wanted
%\phone{phone (optional)}                      % optional, remove the line if not wanted
%\fax{fax (optional)}                          % optional, remove the line if not wanted
\email{willard.yuan@gmail.com}                 % optional, remove the line if not wanted
\homepage{yongyuan.name/cn}                       % optional, remove the line if not wanted
%\extrainfo{♂34岁}                             % optional, remove the line if not wanted
\photo[64pt]{qr}                               % '64pt' is the height the picture must be resized to and 'picture' is the name of the picture file; optional, remove the line if not wanted
\social[github]{willard-yuan}
\quote{\textit{求职意向:图像处理工程师}}

% to show numerical labels in the bibliography; only useful if you make citations in your resume
%\makeatletter
%\renewcommand*{\bibliographyitemlabel}{\@biblabel{\arabic{enumiv}}}
%\makeatother

% bibliography with mutiple entries
%\usepackage{multibib}
%\newcites{book,misc}{{Books},{Others}}

%\nopagenumbers{}                             % uncomment to suppress automatic page numbering for CVs longer than one page
%----------------------------------------------------------------------------------
%            content
%----------------------------------------------------------------------------------
\begin{document}
\maketitle
\vspace{-3em}      %缩小段落的间距

\section{\textbf{教育背景}}
\cventry{2013年 -- 2016年}{硕士学位}{中国科学院大学}{信号与信息专业(研究方向:图像检索)}{保研}{}
\cventry{2009年 -- 2013年}{学士学位}{西安电子科技大学}{电子信息科学与技术专业}{专业 top 3\%}{}
\vspace{-1em}
\section{\textbf{出版物}}
\cventry{2014/04}{\textbf{Yong Yuan}\textnormal{, Xiaoqiang Lu, and Xuelong Li. Learning Hash Functions Using Sparse Reconstruction. ACM ICIMCS, pp. 14-18, 2014(Best Paper Runner-up Award)} }{}{}{}{}
\cventry{2014/06}{\textnormal{朱文涛,}\textbf{袁勇}\textnormal{. Python计算机视觉编程(译作),图灵出版社} }{}{}{}{}
%\cventry{2015.02}{\textnormal{李学龙,卢孝强,}\textbf{袁勇}\textnormal{. 基于潜在语义最小哈希的图像检索的方法(专利)} }{}{}{}{}
\vspace{-1em}

\section{\textbf{科研经历}}
%\subsection{中科院西安光学精密机械研究所(2013--至今)}
\cventry{2013--至今}{基于内容的图像检索(CBIR)}{}{课题研究方向}{}{对海量图像数据,在保持低计算复杂度的前提下,尽可能的提高检索精度进行研究.
\begin{itemize}
\item 熟悉CBIR技术及其检索性能指标评价,掌握了BoW模型、相关反馈等图像检索技术.
\item 掌握了机器学习中一些常用聚类算法、分类方法以及物体识别技术.
\item 熟悉近几年来比较流行的哈希方法,并针对一些流行的和经典的哈希方法进行了性能测试和指标评价,详见HABIR工具包主页:\link[yongyuan.name/habir]{http://yongyuan.name/habir/}.
\item 深入研究过哈希索引方法,提出一种基于稀疏表达的哈希编码方法,发表于ICIMCS14上.
\item 参加过pkbigdata上的图像检索大赛,在近似重复样本搜索上积累了一定的经验,有皮革、纺织图像检索的经历.
\end{itemize}}

\cventry{2015/01--2015/04}{基于卷积神经网络的CBIR演示原型系统PicSearch}{}{兴趣驱动型项目}{}{PicSearch是一个在线图像检索演示系统,使用了在imageNet上已训练好的CNN模型提取图像特征,对于相似语义图像搜索能取得非常满意的检索效果.
\begin{itemize}
\item 线下完成图像特征的提取,在线特征匹配与排序用python实现,服务器采用了基于python的web开发框架CherryPy,前端用了Boostrap框架.
\item 图库为包含29780张图片的Caltech-256公开数据集,代码经过一定的优化后,能及时地响应用户的查询请求 (毫秒级),在线演示地址PicSearch: \link[search.yongyuan.name]{http://search.yongyuan.name/}.
\end{itemize}}

\cventry{2015/03--至今}{基于词袋模型的物体检索原型DupSearch}{}{兴趣驱动型项目}{}{DupSearch是一个针对object retrieval或duplicate search而写的图像检索系统.
\begin{itemize}
\item 在oxford building公开数据库上平均检索精度达到83.35\%,在不复杂化现有模型情况下仍有改进提高mAP 的空间.
\item 图像库测试规模达到15万,并且获得了很不错的检索效果,算法原型系统已被某公司买下,15万衣服库检索效果详见\href{https://github.com/willard-yuan/pkbigdata-image-search}{GitHub}.
\end{itemize}}

\cventry{2014/07--2015/05}{复杂低空飞行的自主避险理论与方法研究(973)}{}{项目参与者}{}{多源协同感知周围环境,对复杂低空环境中可能的危险障碍物进行实时检测,并完成飞行器的自主避险.
\begin{itemize}
\item 负责可见光传感器数据与激光雷达传感器点云数据的融合,消除高压线检测时的误检.
\item 负责桥梁、高压线塔、作为异常目标入侵的滑翔机等危险障碍物的实时检测.
\item 使用了opencv、dlib等计算机视觉开源库,非电力线类障碍物检测采用HOG+SVM物体检测方法.
\end{itemize}}

\vspace{-1em}

\section{\textbf{开源项目}}
\cvline{2013/02--Now}{整理并实现一些流行典型的哈希算法及多种指标评价,目前该Matlab工具包已更新至V2.0,详见\href{https://github.com/willard-yuan/hashing-baseline-for-image-retrieval}{GitHub.}}
\cvline{2013/12--2014/06}{翻译《Programming Computer Vision with Python》时,为使读者更易于理解书中的内容,重新对书上的代码做了整理,并放在github 上,详见\href{http://yongyuan.name/pcvwithpython/}{项目主页.}}
\cvline{2014/02--2014/05}{基于稀疏重构的哈希编码方法的matlab代码及检索指标评价,详见\href{https://github.com/willard-yuan/sparse-reconstruction-hashing}{GitHub.}}
\vspace{-1em}

\section{\textbf{IT技能}}
\cvline{语言}{\small 精通Python, Matlab, 会C/C++, OpenCV(主要是Python接口), Linux, Django, HTML, CSS}
\cvline{GitHub}{\small {\link[github.com/willard-yuan]{https://github.com/willard-yuan}}, 活跃}
\vspace{-1em}      %缩小段落的间距

\section{\textbf{奖项}}
%\cventry{2014.07}{\textnormal {ICIMCS14最佳论文提名奖}}{}{}{}{}
%\cventry{2011--2012}{\textnormal {国家奖学金}}{}{}{}{}
%\cventry{2010--2011}{\textnormal {校内一等奖学金}}{}{}{}{}
%\cventry{2009--2010}{\textnormal {国家励志奖学金}}{}{}{}{}
\cvlistdoubleitem{Best Paper Runner-up Award(2014)}{国家奖学金(2012)}
\cvlistdoubleitem{校内一等奖学金(2010)}{国家励志奖学金(2009)}
\vspace{-1em}      %缩小段落的间距

\section{\textbf{语言}}
\cvline{英语}{\small CET-6和\small CET-4,具备阅读专业英文文献、写作及翻译能力,英文博客翻译曾见诸于CSDN云计算首页。}
%\cvlanguage{英语}{具备阅读专业英文文献能力及写作能力}{CET-6}
\vspace{-1em}      %缩小段落的间距

\section{\textbf{其他}}
\cvline{\hello}{靠谱,责任心强,乐于助人;热爱计算机视觉与互联网;轻度代码洁癖;具有较好的人际沟通、协调和组织能力。}
%\cvlistitem{}

% Publications from a BibTeX file without multibib\renewcommand*{\bibliographyitemlabel}{\@biblabel{\arabic{enumiv}}}% for BibTeX numerical labels
%\nocite{*}
%\bibliographystyle{plain}
%\bibliography{publications}       % 'publications' is the name of a BibTeX file

% Publications from a BibTeX file using the multibib package
%\section{Publications}
%\nocitebook{book1,book2}
%\bibliographystylebook{plain}
%\bibliographybook{publications}   % 'publications' is the name of a BibTeX file
%\nocitemisc{misc1,misc2,misc3}
%\bibliographystylemisc{plain}
%\bibliographymisc{publications}   % 'publications' is the name of a BibTeX file

\end{document}


%% end of file `template_en.tex'.
